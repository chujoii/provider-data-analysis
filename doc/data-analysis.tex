\documentclass[unicode, 12pt, a4paper,oneside,fleqn]{article}


\usepackage{ifxetex}                      %% Для сборки документа и pdflatex'ом, и xelatex'ом
\ifxetex
    %% xelatex
    \usepackage{polyglossia}                       %% загружает пакет многоязыковой вёрстки
    \setdefaultlanguage[spelling=modern]{russian}  %% устанавливает главный язык документа
    \setotherlanguage{english}                     %% объявляет второй язык документа
    \defaultfontfeatures{Ligatures={TeX}}          %% свойства шрифтов по умолчанию
    \setmainfont[Ligatures={TeX}]{Old Standard}    %% задаёт основной шрифт документа
    \setsansfont{Old Standard}                     %% задаёт шрифт без засечек (но Old Standard с засечками заменить на DejaVu Sans?)
    \setmonofont{DejaVu Sans Mono}                  %% задаёт моноширинный шрифт
\else
    %% pdflatex
    \usepackage{cmap}                     %% Поиск русских  слов в pdf
    \usepackage[T2A]{fontenc}             %% Внутренняя кодировка шрифта
    \usepackage[utf8]{inputenc}           %% Кодировка исходного текста;
    %                                      % new: [utf8] (for XeLaTeX);
    %                                      % unmaintained: [ucs];
    %                                      % old, but work: [utf8x] (load ucs)
    %                                      % koi8-r
    %                                      % можно указать cp866 (Alt-кодировка DOS)
    \usepackage[english,russian]{babel}   %% Поддержка русского текста:
    %                                      % включение русификации, русских и
    %                                      % английских стилей и переносов
\fi  



% \hyphenpenalty=50   % переносы разрешены (значение по умолчанию)
% \hyphenpenalty=10000 % запретить переносы
%\hyphenpenalty=9999

% badness - мера разрежённости строки (badness=0 всё хорошо).
% При разбиении абзаца на строки Latex не может создать такие строки,
% badness которых больше, чем значение параметра \tolerance
%\tolerance=10

% к пределу растяжимости каждой из строк в процессе разбиения абзаца
% на строки и вычислений соответствующих значений badness прибавляется
% значение \emergencystretch. Чем больше значение параметра
% \emergencystretch, тем более разреженные строки появятся на печати.
% например: \emergencystretch=10pt конфигурирует TeX для использования
% не более 10pt дополнительного пробела для каждой строки
%\emergencystretch=10pt

% запрет заездов на правое поле документа и переносы. it tells
% TeX to not even look for hyphenation positions
%\pretolerance=9999

% Only words containing at least 16 characters will now be hyphenated.
%\lefthyphenmin=4  % least 4 characters before the hyphen
%\righthyphenmin=4 % least 4 characters after the hyphen


% запрещение вдов ("widow" - короткая строка или слово в конце абзаца)
\widowpenalty=10000
% запрещение сирот ("orphan" - одиночная строка вылетевшая на следующую страницу или колонку)
\clubpenalty=10000
%
% unknown:
%\brokenpenalty=4991
%\predisplaypenalty=10000
%\postdisplaypenalty=1549
%\displaywidowpenalty=1602

\usepackage{graphics}
\usepackage{wrapfig}
\usepackage{multicol}
\usepackage{multirow}
%\usepackage{fullpage}
\usepackage{textcomp} % типографские значки
\usepackage{mathtext} % если нужны русские буквы в формулах
\usepackage{gensymb}  % для спец знаков
\usepackage{amsmath} % для спец знаков в формулах
\usepackage{amssymb} % для спец знаков в формулах
\usepackage{topcapt} % подписи к таблицам
\usepackage{dcolumn} % выравнивание чисел
\usepackage{ulem} % подчёркивание

\usepackage{makeidx} % индекс

\usepackage{import} % for converted svg to pdf

% вращение 
\usepackage{lscape}     % for %\begin{landscape} ...   %\end{landscape}
\usepackage{rotating}   % for sideways and \rotatebox{-90}{}

% водяные знаки
\usepackage[firstpage]{draftwatermark}
\SetWatermarkScale{0.7}
\SetWatermarkLightness{0.9}
\SetWatermarkText{\textbf{Рабочая версия}}

\usepackage[iso,english]{isodate}
\usepackage{datetime2} % print datetime in GOST ISO 8601-2001

\frenchspacing

\usepackage{fancyvrb} % for verbatim text

\usepackage{listings} % for source code
%\lstloadlanguages{lisp}
\lstset{
  language=C,
  %basicstyle=\tiny, %or \small or \footnotesize etc.
  extendedchars=\true, % for russian characters in comments
  %texcl,  % for spaces in russian characters in comments
  keepspaces = true, % for spaces in russian characters in comments. it's work!
  escapechar=|,
  frame=single,
  commentstyle=\itshape,
  inputencoding=utf8,
  stringstyle=\bfseries
}

\usepackage{csvsimple}

% fixme: remove before release
\usepackage[colorinlistoftodos,prependcaption,textsize=tiny]{todonotes}
\setlength{\marginparwidth}{3cm}


\usepackage[colorlinks=true]{hyperref} % url hyperlink (beamer already include it, so move here for prevent conflict)
\definecolor{lightgrey}{RGB}{192, 192, 192}

\author{Роман Приходченко}

\title{Анализ данных}


\makeindex



\begin{document}

% меняем английские термины на русские
\renewcommand\bibname{СПИСОК ЛИТЕРАТУРЫ}
\renewcommand\refname{\centering Список литературы}
\renewcommand\contentsname{\centering Содержание}


% образцы переноса сложных слов - не работает?
% \hyphenation{веб=-ин-тер-фей-се веб-ин-тер-фей-с}
% or use in text: веб"=интерфейс (require: \usepackage[russian]{babel})

% печатаем титульный лист
\makeatletter % generate \@title, \@date, ...
\maketitle

\begin{table}[ht]
  \begin{tabular}{cc}
    \includegraphics[width=2cm]{../CC_BY-SA_88x31.png} &
    \shortstack{документ распространяется в соответствии с
      условиями\\
      \href{http://creativecommons.org/licenses/by-sa/3.0/}{Attribution-ShareAlike} \\
      (Атрибуция — С сохранением условий) CC BY-SA \\
      Копирование и распространение приветствуется.}
  \end{tabular}
\end{table}

\newpage
% печатаем оглавление
\tableofcontents

\newpage

\section{Обработанные данные}
Данные обрабатываются с помощью программы, написанной на языке Python.

Для обработки данных из CSV файлов необходимо выполнить следующие
команды:
\begin{Verbatim}
  ./stat.py ../data/city-001.csv > 1.txt
  ./stat.py ../data/city-002.csv > 2.txt
  paste 1.txt 2.txt
или
  diff -w -B --minimal -y -W 150 1.txt 2.txt
\end{Verbatim}
Знак минус <<->> в выводе данных обозначает, что услуга не подключена.

Результат выполнения:
% fixme: use table
\begin{Verbatim}[commandchars=@\{\}]
@colorbox{lightgrey}{Город №1}                                  @colorbox{lightgrey}{Город №2}

медианный чек = 1200                       медианный чек = 720
средний чек   = 1175 ± 18.97               средний чек   = 770 ± 6.44

@colorbox{lightgrey}{пользователей интернет:}                   @colorbox{lightgrey}{пользователей интернет:}
Ростелеком                   32            Ростелеком                   91 
-                            25            -                            26 
Мегафон                       9            Транстелеком                 10 
Мега Линк                     3            Инфотелеком                  10 
МТС                           3            мобильный интернет            2 
Теле2                         2            Теле2                         1 
Билайн                        1            модем                         1 

@colorbox{lightgrey}{доли рынка интернет (%):}                  @colorbox{lightgrey}{доли рынка интернет (%):}
Ростелеком                42.67%           Ростелеком                64.54% 
-                         33.33%           -                         18.44% 
Мегафон                   12.00%           Транстелеком               7.09% 
Мега Линк                  4.00%           Инфотелеком                7.09% 
МТС                        4.00%           мобильный интернет         1.42% 
Теле2                      2.67%           Теле2                      0.71% 
Билайн                     1.33%           модем                      0.71% 






@colorbox{lightgrey}{проникновение интернет услуг:}             @colorbox{lightgrey}{проникновение интернет услуг:}
Ростелеком                 0.43            Ростелеком                 0.65 
-                          0.33            -                          0.18 
Мегафон                    0.12            Транстелеком               0.07 
Мега Линк                  0.04            Инфотелеком                0.07 
МТС                        0.04            мобильный интернет         0.01 
Теле2                      0.03            Теле2                      0.01 
Билайн                     0.01            модем                      0.01 

@colorbox{lightgrey}{телезрителей:}                             @colorbox{lightgrey}{телезрителей:}
Триколор                     48            Ростелеком                   67 
антенна                      10            Инфотелеком                  30 
Ростелеком                    8            кабельное ТВ                 22 
-                             7            антенна                      10 
Мегафон                       1            -                             5 
спутниковое ТВ                1            Триколор                      4 
                                           спутниковое ТВ                2 
                                           Транстелеком                  2 

@colorbox{lightgrey}{доли рынка ТВ(%):}                         @colorbox{lightgrey}{доли рынка ТВ(%):}
Триколор                  64.00%           Ростелеком                47.18% 
антенна                   13.33%           Инфотелеком               21.13% 
Ростелеком                10.67%           кабельное ТВ              15.49% 
-                          9.33%           антенна                    7.04% 
Мегафон                    1.33%           -                          3.52% 
спутниковое ТВ             1.33%           Триколор                   2.82% 
                                           спутниковое ТВ             1.41% 
                                           Транстелеком               1.41% 

@colorbox{lightgrey}{проникновение ТВ услуг:}                   @colorbox{lightgrey}{проникновение ТВ услуг:}
Триколор                   0.64            Ростелеком                 0.48 
антенна                    0.13            Инфотелеком                0.21 
Ростелеком                 0.11            кабельное ТВ               0.16 
-                          0.09            антенна                    0.07 
Мегафон                    0.01            -                          0.04 
спутниковое ТВ             0.01            Триколор                   0.03 
                                           спутниковое ТВ             0.01 
                                           Транстелеком               0.01 


                                           
@colorbox{lightgrey}{медиана стоимости услуг:}                  @colorbox{lightgrey}{медиана стоимости услуг:}
Триколор                1200.00            Триколор                1200.00 
спутниковое ТВ          1200.00            Ростелеком               700.00 
МТС                      650.00            спутниковое ТВ           650.00 
Ростелеком               635.00            модем                    500.00 
Мегафон                  550.00            Транстелеком             400.00 
Мега Линк                550.00            мобильный интернет       350.00 
Теле2                    250.00            Инфотелеком              320.00 
Билайн                   250.00            кабельное ТВ             320.00 
-                          0.00            Теле2                    300.00 
антенна                    0.00            антенна                   29.00 
                                           -                          0.00 
 
@colorbox{lightgrey}{прибыль:}                                  @colorbox{lightgrey}{прибыль:}
Триколор               57200.00            Ростелеком             77610.00 
Ростелеком             20790.00            Инфотелеком            11300.00 
Мегафон                 4840.00            кабельное ТВ            6325.00 
МТС                     1700.00            Транстелеком            4290.00 
Мега Линк               1650.00            Триколор                3600.00 
спутниковое ТВ          1200.00            спутниковое ТВ          1300.00 
Теле2                    500.00            антенна                  975.00 
Билайн                   250.00            мобильный интернет       700.00 
-                          0.00            модем                    500.00 
антенна                    0.00            Теле2                    300.00 
                                           -                          0.00 
\end{Verbatim}

\newpage
\section{Интерпретация}
Странно, что в опросе практически не присутствуют мобильные операторы
сотовой связи. По моим наблюдениям, молодёжь, даже при наличии
стационарного компьютера или ноутбука, предпочитает смартфон для
интернета, социальных сетей, просмотра коротких видео, музыки.

Вывод: опрос проводился в основном у старшего поколения, и
соответственно получилась нерелевантная выборка.

\section{Анализ}
\begin{enumerate}
\item Люди, использующие только телевизор, скорее всего, не согласятся
  что-либо менять. На изменения они пойдут, вероятно, лишь под влиянием:
  \begin{itemize}
  \item соседей (тоже пенсионеров)
  \item внуков (хочу мультики про пони!)
  \item существенного ухудшения качества приёма
  \item увеличения цены на услуги
  \item поломки старого оборудования
  \item отсутствия интересной передачи (сериала) у текущего провайдера
  \end{itemize}
\item На смену оператора толкают в основном следующие факторы и события:
  \begin{itemize}
  \item качество (картинки ТВ, скорость соединения, \ldots)
  \item проблемы (для решения которых требуются регулярные звонки в
    службу поддержки)
  \item стоимость услуг (самые большие цены у Ростелеком и Триколор)
  \item стоимость подключения
  \item советы знакомых, соседей, сослуживцев
  \item технологии (у некоторых провайдеров для организации
    видеоконференции необходимо подать письменное заявление, которое
    будет рассматриваться неделю)
  \end{itemize}
\item Если предположить, что численность населения пропорциональна
  числу опрошенных, то первый город в два раза проигрывает в
  потенциальной возможности роста числа абонентов сети:

  Число строк файла данных для каждого города (количество опрошенных
  абонентов будет на 1 меньше из-за заголовка файла CSV) можно
  получить так:
  \begin{Verbatim}[commandchars=@\{\}]
    @textcolor{green}{%} wc --lines data/city-*.csv@colorbox{green}{ @strut}
     76 city-001.csv
    142 city-002.csv
  \end{Verbatim}
\end{enumerate}
\section{Выводы}
В рассмотренных городах преобладают Ростелеком и Триколор, у которых
самые большие цены на услуги связи и телевидения.

Появление конкурентов с приемлемыми ценами и отличным качеством услуг
приведёт к перераспределению абонентов.

  \begin{itemize}
  \item Город №1:\\
    Малочисленное население, использующее в основном телевизор (у трети населения нет интернета).
    Небольшие перспективы для развития сети.
  \item Город №2:\\
    Перспективный город для развития сети.
  \end{itemize}


\end{document}
